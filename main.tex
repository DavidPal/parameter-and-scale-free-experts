\documentclass[12pt]{article}

\usepackage{fullpage,amssymb,amsmath,amsthm}

\newcommand{\R}{\mathbb{R}}
\newcommand{\norm}[1]{\left\|#1\right\|}
\newcommand{\KL}[2]{D\left(#1 \middle\| #2 \right)}
\DeclareMathOperator{\Regret}{Regret}

\begin{document}

\title{Parameter-Free and Scale-Free Algorithm for Learning with Expert Advice}
\author{Francesco Orabona \and D\'avid P\'al}
\maketitle

\section{Introduction}

Learning with Expert Advice (LEA) is an online problem where in each round an
algorithm chooses a point $x_t$ in the probability simplex $\Delta_N = \{x \in
\R^N ~:~ x \ge 0, \norm{x}_1 = 1 \}$ and as a response it receives a loss
vector $\ell_t \in \R^N$ and suffers a loss $\langle \ell_t, x_t \rangle$.
Algorithm's goal is to be competitive with a strategy that in each round
chooses the same point $u \in \Delta_N$.
The difference of their losses, after $T$ rounds, is called
\emph{regret with respect to u}. Formally,
$$
\Regret_T(u) = \sum_{t=1}^T \langle \ell_t, x_t \rangle - \sum_{t=1}^T \langle \ell_t, u \rangle \; .
$$

We wish to construct an algorithm that receives as an input \emph{prior
distribution} $\pi \in \Delta_N$ and an positive integer $N$ and satisfies
\begin{equation}
\label{equation:parameter-scale-free-bound}
\forall T \ge 0 \quad \forall u \in \Delta_N \qquad \Regret_T(u) \le C \sqrt{(1 + \KL{u}{\pi}) \sum_{t=1}^T \norm{\ell_t}_\infty^2} \; .
\end{equation}
for some universal constant $C > 0$.  In particular, the algorithm does
\emph{not} receive neither $u$, $T$, $\sum_{t=1}^T \norm{\ell_t}_\infty^2$ nor
$\max_{1 \le t \le T} \norm{\ell_t}_\infty$, nor learning rate nor step size.
Thus the algorithm must be \emph{parameter-free} and \emph{scale-free}.

The bound \eqref{equation:parameter-scale-free-bound} is in contrast with
existing algorithms for LEA. Existing algorithms either assume that
$\norm{\ell_t}_\infty \le 1$ for all $t \ge 0$ or the bound on their regret is
looser, e.g., $O\left( \sqrt{\ln N \sum_{t=1}^T \norm{\ell_t}_\infty^2}
\right)$.

We will use the following facts. For any $\pi \in \Delta_N$, the function
$f_\pi(u) = \KL{u}{\pi}$ is $1$-strongly convex with respect to
$\norm{\cdot}_1$; this result is called Pinsker's inequality.

\section{Parameter- and Scale-Free Algorithms}

We design a parameter- and scale-free algorithm using a two layer construction.
At the bottom layer there are infinitely many ``slave'' algorithms indexed by
positive integers $i=1,2,\dots$. At the top layer there is a single ``master''
algorithm that combines predictions of the slave algorithms.  Both slave and
master algorithms are instances of AdaHedge. The only technical part is that
the master is solving LEA with infinitely many experts.


For any $\pi \in \Delta_N$ and any $\lambda > 0$, AdaHedge with regularizer
$R(u) = \lambda \cdot \KL{u}{\pi}$ satisfies the following regret bound
$$
\forall T \ge 0 \quad \forall u \in \Delta_N \qquad \Regret_T(u) \le \sqrt{3}\max\left\{2, \frac{1}{\sqrt{2 \lambda}}\right\} \sqrt{\sum_{t=1}^T \norm{\ell_t}_\infty^2} \left(1 + \lambda \cdot \KL{u}{\pi} \right) \; .
$$
We need to specify $\lambda$'s and $\pi$'s for slave and master algorithms.
All slaves will use the same prior $\pi$; this prior can be arbitrary.  Slave
algorithm with index $i$ will use $\lambda = \lambda_i = \frac{1}{4} \cdot 2^{-i}$. The master
algorithm will use $\lambda = 1$ and prior distribution $\pi = (2^{-1}, 2^{-2},
2^{-3}, \dots)$.

We denote by $x^{(i)}_t$ the prediction of $i$-th slave algorithm.
We denote by $p_t$ the distribution predicted by master algorithm in round $t$.
The final prediction $x_t \in \Delta_N$ of the combined algorithm is the mixture
$$
x_t = \sum_{i=1}^\infty p_{t,i} \cdot x^{(i)}_t \; .
$$
Note that since both slaves and master are scale-free, the combined algorithm
is scale-free as well. Namely, the sequences $\{x^{(i)}\}_{t=1}^\infty$,
$\{p_t\}_{t=1}^\infty$, $\{x_t\}_{t=1}^\infty$ do not change if the sequence of
loss vectors $\{\ell_t\}_{t=1}^\infty$ is replaced with the sequence $\{c
\ell_t\}_{t=1}^\infty$ for any $c > 0$.

We bound the regret of combined algorithm, by decomposing the regret
across the two layers. For any $T \ge 0$, any $i \ge 1$ and any $u \in \Delta_N$, we have
\begin{align*}
\Regret_T(u)
& = \sum_{t=1}^T \langle \ell_t, x_t \rangle - \sum_{t=1}^T \langle \ell_t, u \rangle \\
& = \underbrace{\left(\sum_{t=1}^T \langle \ell_t, x_t \rangle - \sum_{t=1}^T \langle \ell_t, x^{(i)}_t \rangle \right)}_{\Regret^{\text{master}}_T(e_i)}
+ \underbrace{\left(\sum_{t=1}^T \langle \ell_t, x^{(i)}_t \rangle - \sum_{t=1}^T \langle \ell_t, u \rangle \right)}_{\Regret^{(i)}_T(u)} \\
& = \underbrace{\left(\sum_{t=1}^T \sum_{t=1}^T p_{t,i} \cdot \langle \ell_t, x^{(i)} \rangle - \sum_{t=1}^T \langle \ell_t, x^{(i)}_t \rangle \right)}_{\Regret^{\text{master}}_T(e_i)}
+ \underbrace{\left(\sum_{t=1}^T \langle \ell_t, x^{(i)}_t \rangle - \sum_{t=1}^T \langle \ell_t, u \rangle \right)}_{\Regret^{(i)}_T(u)} \\
& \le \sqrt{3} \max\left\{2, \frac{1}{\sqrt{2}} \right\} \sqrt{\sum_{t=1}^T \norm{\ell_t}_\infty^2} (1 + \ln(2^i)) \\
& \qquad + \sqrt{3} \max\left\{2, \frac{1}{\sqrt{\frac{1}{2} 2^{-i}}} \right\} \sqrt{\sum_{t=1}^T \norm{\ell_t}_\infty^2} (1 + 2^{-i} \cdot \KL{u}{\pi}) \\
& \le \sqrt{12 \sum_{t=1}^T \norm{\ell_t}_\infty^2} (1 + i) + \sqrt{6 \sum_{t=1}^T \norm{\ell_t}_\infty^2} \left(\sqrt{2^i} + \frac{\KL{u}{\pi}}{\sqrt{2^{i}}} \right) \; .
\end{align*}
Since the bound holds for any $i$, we can choose an optimal value for it.
We choose $i$ so that $2^i \approx \KL{u}{\pi}$. More precisely, we choose
$$
i = \max\left\{1, \left\lceil  \log_2 \KL{u}{\pi} \right\rceil \right\} \; .
$$
Note that
$$
\max\{1, \log_2 \KL{u}{\pi}\} \le i < \max\{1, 1 + \log_2 \KL{u}{\pi}\}
$$
and also $i \le 1 + \sqrt{\KL{u}{\pi}}$ which follows from the inequality
$\max\{1, \lceil \log_2 x \rceil\} \le 1 + \sqrt{x}$ valid for all $x \ge 0$.

With this choice of $i$, the regret of the combined algorithm is
\begin{align*}
\Regret_T(u)
& \le \sqrt{12 \sum_{t=1}^T \norm{\ell_t}_\infty^2} (1 + i) + \sqrt{6 \sum_{t=1}^T \norm{\ell_t}_\infty^2} \left(\sqrt{2^i} + \frac{\KL{u}{\pi}}{\sqrt{2^{i}}} \right) \\
& \le \sqrt{12 \sum_{t=1}^T \norm{\ell_t}_\infty^2} (2 + \sqrt{\KL{u}{\pi}}) \\
& \qquad + \sqrt{6 \sum_{t=1}^T \norm{\ell_t}_\infty^2} \left(\sqrt{2^{\max\{1, 1 + \log_2 \KL{u}{\pi}\}}} + \frac{\KL{u}{\pi}}{\sqrt{2^{\max\{1, \log_2 \KL{u}{\pi} \}}}} \right) \\
& \le \sqrt{12 \sum_{t=1}^T \norm{\ell_t}_\infty^2} (2 + \sqrt{\KL{u}{\pi}}) + \sqrt{6 \sum_{t=1}^T \norm{\ell_t}_\infty^2} \max \left\{2 + \frac{2}{\sqrt{2}}, (\sqrt{2} + 1) \sqrt{\KL{u}{\pi}} \right\} \; .
\end{align*}
The last inequality can be proved by considering the cases $\KL{u}{\pi} \le 2$
and $\KL{u}{\pi} > 2$.

\end{document}
